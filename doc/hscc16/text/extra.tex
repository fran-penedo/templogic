\clearpage
\section{Look-up table computation of robustness}
\begin{proposition}
\label{th:primitives-props}
Let $s$ be an $n$-dimensional signal. The following hold:
\begin{enumerate}
    \item the sets of primitives $\CA{P}_1$ and $\CA{P}_2$ are
    closed under negation, i.e. $\phi \in \CA{P} \Implies \notltl\phi \in \CA{P}$,
    where $\CA{P}$ is $\CA{P}_1$ or $\CA{P}_2$;
    \item robustness  of 1$^{st}$ order primitives:
    \begin{align}
    	r(s, \Event_{[\tau_1, \tau_2)} (s_i \leq \mu)) &= \inf_{t \in [\tau_1, \tau_2)} \{ s_i(t) \} - \mu \\
    	r(s, \Always_{[\tau_1, \tau_2)} (s_i \leq \mu)) &= -\inf_{t \in [\tau_1, \tau_2)} \{ -s_i(t) \} - \mu
    \end{align}
    \item robustness  of 2$^{nd}$ order primitives:
    \begin{align}
    	r(s, \varphi_{\Event\Always}) &= \inf_{t_1 \in [\tau_1, \tau_2)} \big\{  \sup_{t_2 \in [0, \tau_3)} \{ s_i(t) \} \big\} - \mu \\
    	r(s, \varphi_{\Always\Event}) &= -\inf_{t_1 \in [\tau_1, \tau_2)} \big\{  \sup_{t_2 \in [0, \tau_3)} \{ -s_i(t) \} \big\} - \mu
    \end{align}
    where $\varphi_{\Event\Always} = \Event_{[\tau_1, \tau_2)} \Always_{[0, \tau_3)} (s_i \leq \mu)$
    and \quad $\varphi_{\Always\Event} = \Always_{[\tau_1, \tau_2)} \Event_{[0, \tau_3)} (s_i \leq \mu)$.
\end{enumerate}
\end{proposition}
\begin{proof}
The proof follows immediately from the definitions of the eventually and always operators and of the robustness.
\end{proof}




{\color{orange}
TODO: informal description. If we include this
in the paper, we need to massage it.
NOTE: I think we can remove the algorithms
from this version of the paper
(comment to latex code please) and just
briefly explain how it works.

If we consider piecewise constant (or linear)
signals obtained by uniform sampling, i.e.
at constant rate, then we can speed-up
robustness computation by pre-computing
lookup tables for each signal before the
inference algorithm is executed.
For each signal $s$, we compute the robustness
degree of $s$ and $-s$ for every possible
valuation of the 2 and 3 time parameters
of primitives in $\CA{P}_1$ and $\CA{P}_2$,
respectively. The space parameter $\mu$ is set
to 0.
By Prop.~\ref{th:primitives-props}, the computation
of robustness reduces to a lookup in a table,
the subtraction of $\mu$ and a possible change of sign.

For the set of primitives $\CA{P}_2$, the lookup
tables can efficiently be computed as
shown in Alg.~\ref{alg:lkt-1d}, Alg.~\ref{alg:lkt-nd}
and Alg.~\ref{alg:lkt-prim}. The robustness is
computed usinf the lookup table in
Alg.~\ref{alg:robustness-lookup}.

The size of the lookup tables for all signals $S$ is
$2\times \card{S} \times n \times T^3$, where $n$ is the
dimension of the signals, $T$ is the number
of samples per signal, and 2 in the expression
indicates that we compute two lookup tables per
signal, one for each type of primitive.

Another approach is to save the entries of the
lookup table as these are computed during the
optimization procedure. In this way, later
operations will not have to recompute these.
This approach is in line with lazy-computation
heuristics.
}

\begin{algorithm}
\caption{Lookup table for 1d signal -- $lkt\_1d(\cdot)$}
\label{alg:lkt-1d}
\DontPrintSemicolon
\KwIn{$s$ -- signal}
\KwOut{Lookup table}
\BlankLine

$T \asgn length(s)$ \tcp*[f]{number of samples per signal}\;
$lk1 \asgn fill((T, T), \infty)$\;
$lk1[0, 0:T] \asgn s$\;
\For{$i = 1 \text{ to } T$}{
  $n \asgn T - i$\;
  $lk1[i, 0:n] \asgn pmin(lk1[i-1, 0:n], lk1[i-1, 1:n+1])$\;
}
$lk \asgn fill((T, T, T), -\infty)$\;
\For{$\tau_3 = 0 \text{ to } T$}{
  $n \asgn T - \tau_3$\;
  $lk[0:n, 0, \tau_3] \asgn lk1[\tau_3, 0:n]$\;
  \For{$j = 1\text{ to } n$}{
    $m \asgn n - j$\;
    $lk[0:m, j, \tau_3] \asgn pmax(lk[0:m, j-1, \tau_3],$\;
    \qquad\qquad\qquad\qquad\qquad\  $lk[1:m+1, j-1, \tau_3])$\;
  }
}
\Return{$lk$}
\end{algorithm}

\begin{algorithm}
\caption{Lookup table for n-dimensional signal -- $lkt\_nd(\cdot)$}
\label{alg:lkt-nd}
\DontPrintSemicolon
\KwIn{$s$ -- signal}
\KwOut{Lookup table}
\BlankLine

$lk \asgn fill((n, T, T, T), -\infty)$\;
\For{$i = 0 \text{ to } n$}{
  $lk[i, 0:T, 0:T, 0:T] \asgn lkt\_1d(s_i)$
}
\Return{$lk$}
\end{algorithm}

\begin{algorithm}
\caption{Lookup tables -- $lkt\_prim(\cdot)$}
\label{alg:lkt-prim}
\DontPrintSemicolon
\KwIn{$S$ -- set of signals}
\KwOut{Lookup table}
\BlankLine

$lk\_max\_min \asgn fill((\card{S}, n, T, T, T), 0)$\;
\For{$s \in S$}{
  $lk\_max\_min[s, 0:n, 0:T, 0:T, 0:T] \asgn lookup\_nd(s)$
}
$lk\_min\_max \asgn fill((\card{S}, n, T, T, T), 0)$\;
\For{$s \in S$}{
  $lk\_min\_max[s, 0:n, 0:T, 0:T, 0:T] \asgn -lookup\_nd(-s)$
}
\Return{$(lk\_max\_min, lk\_min\_max)$}
\end{algorithm}

\begin{algorithm}
\caption{Robustness -- $robustness(\cdot)$}
\label{alg:robustness-lookup}
\DontPrintSemicolon
\KwIn{$s$ -- signal}
\KwIn{$\phi$ -- primitive in $\CA{P}_2$}
\KwIn{$(lk\_max\_min, lk\_min\_max)$ -- lookup tables}
\KwOut{robustness}
\BlankLine

$i_1, i_2, i_3 \asgn \lfloor\tau_1\rfloor, \lceil\tau_2 - \tau_1\rceil, \lceil \tau_3\rceil$

\uIf{$F_{[\tau_1, \tau_2]} G_{[0, \tau_3]} (s^j \leq \mu)$}{
  \Return{$lk\_max\_min[s, j, i_1, i_2, i_3] - \mu$}
}
\uElseIf{$G_{[\tau_1, \tau_2]} F_{[0, \tau_3]} (s^j > \mu)$}{
  \Return{$\mu - lk\_max\_min[s, j, i_1, i_2, i_3]$}
}
\uElseIf{$G_{[\tau_1, \tau_2]} F_{[0, \tau_3]} (s^j \leq \mu)$}{
  \Return{$lk\_min\_max[s, j, i_1, i_2, i_3] - \mu$}
}
\ElseIf{$F_{[\tau_1, \tau_2]} G_{[0, \tau_3]} (s^j > \mu)$}{
  \Return{$\mu - lk\_min\_max[s, j, i_1, i_2, i_3]$}
}

%def robustness_lookup(t0, t1, t3, mu, sindex, pred_less, ev_al, traces):
%    '''Robustness lookup for primitives
%    F_{[t_0, t_1]} (G_{[0, t_3]} (s_{sindex} \leq mu)), if ev_al is true,
%    G_{[t_0, t_1]} (F_{[0, t_3]} (s_{sindex} \leq mu)), otherwise.
%    If pred_less is False then the predicate is assumed to be negated, i.e. gt.
%    '''
%    t0, t1, t3 = np.floor(t0), np.ceil(t1), np.ceil(t3)
%    sgn = 1 if pred_less else -1
%    if ev_al:
%        return sgn*(traces.traces_lkt_max_min[:, sindex, t0, t1-t0, t3] - mu)
%    return sgn*(traces.traces_lkt_min_max[:, sindex, t0, t1-t0, t3] - mu)
\end{algorithm}

%\subsection{Figures}
%
%\begin{figure}
%\centering
%\epsfig{file=fly.eps}
%\caption{A sample black and white graphic (.eps format).}
%\end{figure}
%
%\begin{figure}
%\centering
%\epsfig{file=fly.eps, height=1in, width=1in}
%\caption{A sample black and white graphic (.eps format)
%that has been resized with the \texttt{epsfig} command.}
%\end{figure}
%
%\begin{figure}
%\centering
%\psfig{file=rosette.ps, height=1in, width=1in,}
%\caption{A sample black and white graphic (.ps format) that has
%been resized with the \texttt{psfig} command.}
%\end{figure}
%
%\begin{figure*}
%\centering
%\epsfig{file=flies.eps}
%\caption{A sample black and white graphic (.eps format)
%that needs to span two columns of text.}
%\end{figure*}
