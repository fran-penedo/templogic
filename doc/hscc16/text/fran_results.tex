\subsection{Fran's results}
\begin{algorithm}
\caption{Tree to formula -- $Tree2STL(\cdot)$}
\label{alg:tree2formula}
\DontPrintSemicolon
\KwIn{$node$ -- Starting node of a tree}
\KwOut{Formula}
\BlankLine

\uIf{$node$ is $leaf$}{
    \Return{$\True$}
}

$\phi \asgn$ formula associated with $node$

\uIf{$node.right$ is $leaf$}{
    \Return{$\phi \andltl Tree2STL(node.left)$}
}\Else{
    \Return{$(\phi \andltl Tree2STL(node.left)) \orltl (\notltl \phi \andltl Tree2STL(node.right))$}
}
\end{algorithm}

We implemented and tested two different instances of Alg.\ref{alg:inf}, $I_1$ and $I_2$, defined by the choice of parameters given in Table \ref{tab:inst}. In the case of $I_1$, the implementation was done in Matlab using standard libraries, using the simulated annealing optimization method and run on a 3.5 GHz processor with 16 GB RAM. As for $I_2$, we used the SciPy library for Python, solving the optimization problem with its implementation of the differential evolution algorithm, and we tested it on similar hardware.

{\color{blue} Review $I_1$}

\begin{table}
\begin{tabular}{|c|c|c|c|}
    \hline
    Instance & Primitives & Impurity & Stopping \\ \hline
    $I_1$ & $\CA{P}_1$ & MG & class rate >0.99 \\ \hline
    $I_2$ & $\CA{P}_2$ & IG & depth >3 \\ \hline
\end{tabular}
\caption{Algorithm parameters}
\label{tab:inst}
\end{table}

Maritime surveillance

We tested the $I_2$ instance using a non stratified 10-fold crosss-validation with a random permutation of the data set, obtaining a mean misclassification rate of 0.007 with a standard deviation of 0.008 and a run time of about 40 hours. A (simplified) sample formula learned from one of the runs is:

\begin{equation}
\begin{aligned}
    \phi = & ( \phi_1 \andltl (\notltl \phi_2 \orltl (\phi_2 \andltl \notltl \phi_3))) \orltl (\notltl \phi_1 \andltl (\phi_4 \andltl \phi_5)) \\
    \phi_1 & = \Always_{[199.70, 297.27)} \Event_{[0.00, 0.05)} (x \le 23.60) \\
    \phi_2 & = \Always_{[4.47, 16.64)} \Event_{[0.00, 198.73)} (y \le 24.20) \\
    \phi_3 & = \Always_{[34.40, 52.89)} \Event_{[0.00, 61.74)} (y \le 19.62) \\
    \phi_4 & = \Always_{[30.96, 37.88)} \Event_{[0.00, 250.37)} (x \le 36.60) \\
    \phi_5 & = \Always_{[62.76, 253.23)} \Event_{[0.00, 41.07)} (y \le 29.90)
\end{aligned}
\end{equation}

We can see in Figure~\ref{fig:navalresults} how the thresholds for $\phi_1$ and $\phi_2$ capture the key features of the data set. Notice also the insight we can gain from their plain English translation: "Normal vessels' $x$ coordinate is below 23.6 during the last 100 seconds, i.e., they approach and remain at the port" and "normal vessels' $y$ coordinate never go below 24.2, i.e., they don't approach the island". As usual when employing decision trees, deeper formulae focus on finer details of the data set.

{\color{blue} The translations are more level 1 than 2, so maybe do it for the level 1 sample? The first interval is kind of short even for $\phi_2$, so it's justified, but seems weird.}
