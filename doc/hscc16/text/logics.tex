\section{Signal Temporal Logic}\label{sec:stl}

Let $\BB{R}$ be the set of real numbers and $t\in \BB{R}$, we denote the interval $[t, \infty)$ by $\BB{R}_{\geq t}$.
%Let $a\leq b \in \BB{Z}$, we denote the set $\{a, \ldots, b\}$ by $\overline{(a, b)}$.

Let $\CA{S} = \{s :\BB{R}_{\geq 0} \to \BB{R}^n\}$ be the set of all continuous parameterized curves in the $n$-dimensional Euclidean space $\BB{R}^n$.
In this paper, an element of $\CA{S}$ is called a {\em signal} and its parameter is interpreted as {\em time}.
Given a signal $s \in \CA{S}$, the components of $s$ are denoted by $s_i$, $i \in (1, \dots, n)$. The {\em suffix} at time $t \geq 0$ of a signal is
is denoted by $s[t] \in \CA{S}$ and represents the signal $s$ shifted forward in time by $t$ time units, i.e. $s[t](\tau) = s(\tau+t)$ for all $\tau \in \BB{R}_{\geq 0}$. The set of all functionals over $\BB{R}^n$ is denoted by $\CA{F} = \{ f : \BB{R}^n \to \BB{R} \}$.

The syntax of {\em Signal Temporal Logic} (STL) \cite{maler_monitoring_2004} is defined as follows:
\begin{equation*}
\phi ::= \True \ |\  p_{f, \mu} \ |\ \notltl \phi  \ |\ \phi_1 \andltl \phi_2 \ |\ \phi_1 \Until_{[a, b)} \phi_2
\end{equation*}

where $\True$ is the Boolean {\em true} constant; $p$ is a predicate over $\BB{R}^n$ parameterized by the functional $f \in \CA{F}$ and $\mu \in \BB{R}$
of the form $p_{f, \mu}(x) = f(x) \leq \mu$; $\notltl$ and $\andltl$ are the Boolean operators of negation and conjunction; and $\Until_{[a, b)}$ is the bounded temporal operator {\em until}.

The semantics of STL is defined over signals in $\CA{S}$ and is
defined recursively as follows~\cite{maler_monitoring_2004}:
\begin{align*}
& s[t] \models \True &\Leftrightarrow \quad& \True \\
& s[t] \models p_{f, \mu} &\Leftrightarrow \quad& (p_{f, \mu}(s(t)) = \True) \equiv (f(s(t)) \leq \mu)\\
& s[t] \models \notltl \phi &\Leftrightarrow \quad& \neg (s[t] \models \phi)\\
& s[t] \models (\phi_1 \andltl \phi_2) &\Leftrightarrow \quad& (s[t] \models \phi_1) \wedge (s[t] \models \phi_2)\\
& s[t] \models (\phi_1 \Until_{[a, b)} \phi_2) &\Leftrightarrow \quad& \exists t_u \in [t+a, t+b) \text{ s.t. } \big(s[t_u] \models \phi_2\big)\\
& & & \wedge \big(\forall t_1 \in [t+a, t_u) \ s[t_1] \models \phi_1\big)
\end{align*}

A signal $s\in \CA{S}$ is said to satisfy an STL formula $\phi$ if and only if $s[0] \models \phi$. We extend the type of allowed inequality predicated in STL to $s[t] \models (f(s(t)) > \mu) \equiv s[t] \models \neg \big((f(s(t)) \leq \mu) \big)$. Thus, predicates are parameterized in this paper by a functional $f \in \CA{F}$, a real number $\mu \in \BB{R}$ and an order relation $\sim \in \{\leq, >\}$.
The Boolean operator of disjunction is defined using De~Morgan's laws. Also, the temporal operators {\em eventually} and {\em globally} are defined as $\Event_{[a, b)} \phi \equiv \True \Until_{[a, b)} \phi$ and $\Always_{[a, b)} \phi \equiv \notltl \Event_{[a, b)} \notltl \phi$, respectively.

The {\em language} associated with an STL formula $\phi$ is the set of all signals in $\CA{S}$ which satisfy $\phi$ and is denoted by $\CA{L}(\phi) = \{s \in \CA{S}\ |\ s \models \phi\}$.

In addition to Boolean semantics, STL admits {\em quantitative semantics}~\cite{donze_robust_2010,fainekos_robustness_2009},
which is formalized by the notion of {\em robustness degree}.
The robustness degree of a signal $s\in \CA{S}$ with respect to an STL formula $\phi$ at time $t$ is a functional $r(s, \phi, t)$ and is recursively
defines as
\begin{align*}
&r(s, \True, t) &=\ & r_{{}_\True}\\
&r(s, p_{f, \mu}, t) &=\ & \mu - f(s(t))\\
&r(s, \notltl \phi, t) &=\ & -r(s, \phi, t)\\
&r(s, \phi_1 \andltl \phi_2, t) &=\ & \min\{r(s, \phi_1, t), r(s, \phi_2, t)\}\\
&r(s, \phi_1 \Until_{[a, b)} \phi_2, t) &=\ &\\
&\quad \sup_{t_u \in [t+a, t+b)} \Big\{\min\big\{  r(s, \phi_2, t_u),\inf_{t_1\in [t+a, t_u)} \{r(s, \phi_1, t_1)\} \big\}  \Big\} \span \span
\end{align*}
where $r_{{}_\True} \in \BB{R}_{\geq 0} \cup \{\infty\}$ is a large constant representing the maximum value of the robustness.
Note that a positive robustness degree $r(s, \phi, 0)$ of a signal $s$ with respect to a formula $\phi$ implies that $s$ satisfies $\phi$. In the following, we denote by $r(s, \phi)$ the robustness degree $r(s, \phi, 0)$
at time $0$.
Robustness can be extended to the derived predicate and operators as follows:
\begin{align*}
&r(s, p_{f, \mu}^>, t) &=\ & f(s(t)) - \mu\\
&r(s, \phi_1 \orltl \phi_2, t) &=\ & \max\{r(s, \phi_1, t), r(s, \phi_2, t)\}\\
&r(s, \Event_{[a, b)} \phi, t) &=\ & \sup_{t_u \in [t+a, t+b)}\{r(s, \phi, t_u)\}\\
&r(s, \Always_{[a, b)} \phi, t) &=\ & \inf_{t_u \in [t+a, t+b)}\{r(s, \phi, t_u)\}\\
\end{align*}
where $p_{f, \mu}^>$ is the derived predicate of the form $f(x) > \mu$.

Moreover, the interpretation of robustness degree as a quantitative measure of satisfaction is justified by the following proposition from~\cite{donze_efficient_2013}.
\begin{proposition}
\label{th:robustness-relative}
Let $s \in \CA{S}$ be a signal and $\phi$ an STL formula such that $r(s, \phi) > 0$. All signals $s' \in \CA{S}$ such that $s' - r(s, \phi) > 0$ satisfy the formula $\phi$, i.e. $s' \models \phi$.
\end{proposition}


{\em Parametric Signal Temporal Logic} (PSTL) was introduced in~\cite{asarin_parametric_2012} as an extension of STL, where formulae are finitely parameterized.
A PSTL formula is similar to an STL formula, however all the time bounds in the time intervals associated with temporal operators and all the constants in the
inequality predicates are replaced by free parameters.
The two types of parameters are called respectively {\em time} and {\em space}.
Specifically, let $\psi$ be a PSTL formula and $n_p$ and $n_{TL}$ be the number of predicates and temporal operators contained in $\psi$. %, respectively.
The parameters space of $\psi$ is $\Theta = \Pi \times T$, where $\Pi\subseteq\BB{R}^{n_p}$ is set of all possible {\em space} parameters
and $T = T_1 \times \ldots T_{n_{TL}}$ is the set of all {\em time} parameters, where $T_i = \{(a_i, b_i) \in \BB{R}_{\geq 0}^2 \ |\  a_i \leq b_i  \}$ for all $i \in (1, \dots, n_{TL})$.
Conversely, let $\psi$ be a PSTL formula, every parameter assignment $\theta\in\Theta$ induces a corresponding STL formula $\phi_\theta$, where all the space and time parameters of $\psi$ have been fixed according to $\theta$. This assignment is also referred to as valuation $\theta$ of $\psi$.
For example, given $\psi = \Always_{[a ,b)} (s_1 \leq c)$ and $\theta=[0,1,2.5]$, we obtain the STL formula $\phi_\theta = \Always_{[0 ,1)} (s_1 \leq 2.5)$.


\begin{comment}%simplified and bounds moved in optimization
Let $\psi$ be a PSTL formula and $n_p$ and $n_{TL}$ be the number of  predicates and temporal operators contained in $\phi$, respectively.
The parameters space of $\psi$ is $\Theta = \Pi \times T$, where $\Pi$ is the compact hyper-box in $\BB{R}^{n_p}$ of all possible {\em space} parameters
and $T = T_1 \times \ldots T_{n_{TL}}$ is the set of all {\em time} parameters, $T_i = \{(a_i, b_i) \in \BB{R}^2 \ |\ \tau^L_i \leq a_i \leq b_i \leq \tau^U_i, \tau^L_i \leq \tau^U_i \in \BB{R}_{\geq 0} \}$ for all $i \in \overline{(1, n_{TL})}$.
Let $\psi$ be a PSTL formula. We denote by $\theta$ the full parameterization of $\psi$, i.e. the vector of all space and time parameters.
A {\em valuation} $v$ of $\psi$ is an assignment of values from $\Theta$ to all parameters $\theta$, i.e. $v : \theta \to \Theta$.
Each valuation $v$ of an PSTL formula $\psi$ induces an STL formula $\phi_v$ where the parameters $\theta$ are replaced by their corresponding values $v(\theta)$.
\end{comment}



